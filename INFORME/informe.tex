%NO MODIFICAR ESTA SECCION!
\documentclass[12pt]{article} % Define la clase del documento, en este caso, un artículo
\usepackage{mathptmx}
\usepackage[letterpaper,margin=3cm]{geometry} % Configura el tamaño del papel y los márgenes del documento
\usepackage{graphicx} % Permite la inserción de imágenes
\usepackage[spanish]{babel}% Activar esta configuración para informes en español, ajusta el idioma del documento
\usepackage[usenames]{color} % Permite el uso de colores definidos por nombre en el documento
\usepackage{hyperref} % Habilita enlaces y referencias dentro del documento
\usepackage{booktabs} % Proporciona comandos para crear tablas de alta calidad
\usepackage{natbib}
\usepackage{tikz} % Permite la creación de gráficos y diagramas vectoriales directamente en LaTeX
\usepackage{float} % Para controlar la posición de los elementos flotantes, como imágenes, con la opción [H]
\usepackage{diagbox} % Permite crear celdas con líneas diagonales en tablas
\usepackage{listings} % Permite la inclusión y formateo de código fuente en el documento
\usepackage{xcolor} % Paquete para definir y usar colores en el documento
\usepackage{parskip} % Añade espacio entre párrafos en lugar de sangrías
\usepackage{fancyhdr} % Permite personalizar encabezados y pies de página
\usepackage{amsmath} % Proporciona una amplia variedad de entornos y comandos matemáticos
\usepackage{microtype}  % Mejora la justificación del texto

\hypersetup{
    colorlinks   = true,    % Colorea los enlaces
    linkcolor    = blue,    % Color de los enlaces internos (por ejemplo, en la tabla de contenidos)
    citecolor    = blue,    % Color de los enlaces a citas bibliográficas
    filecolor    = blue,    % Color de los enlaces a archivos locales
    urlcolor     = blue,    % Color de los enlaces externos (URLs)
}


\geometry{
    paperwidth=21.6cm,  % Ancho del papel
    paperheight=27.9cm,  % Largo del papel
    left=3cm,  % Margen izquierdo
    right=2cm,  % Margen derecho
}

\pagestyle{fancy} % Usa el estilo fancyhdr
\fancyhf{} % Borra todos los encabezados y pies de página
\renewcommand{\headrulewidth}{0pt}
\renewcommand{\footrulewidth}{0pt} % Desactiva la línea horizontal predeterminada en el pie
%\setlength{\headheight}{1.5cm} % Ajusta la altura del encabezado para hacer espacio para la línea
%\fancyhead[L]{\raisebox{0.20cm}{\textbf{Sistemas de Transporte}}} % Añade el texto en la parte izquierda del encabezado, subiéndolo ligeramente
%\fancyhead[R]{\raisebox{0.1cm}{\includegraphics[width=0.25\linewidth]{LOGO_UNIVERSIDAD.jpg}}} % Añade la imagen en la parte derecha del encabezado y súbela un poco
%\fancyhead[C]{\rule{\textwidth}{0.6pt}} % Añade una línea horizontal superior centrada
%\fancyfoot[C]{\rule{\textwidth}{0.6pt}} % Añade una línea horizontal en el pie de página centrada
%\fancyfoot[R]{\raisebox{-1.5\baselineskip}{\thepage}} % Coloca el número de página a la derecha, con suficiente espacio debajo de la línea
%\geometry{top=2.5cm, bottom=2.5cm} % Ajusta los márgenes superior e inferior

\fancyhead[R]{\thepage} % Número de página a la derecha en el encabezado


% Definición de colores al estilo Visual Studio Code
\definecolor{codegreen}{rgb}{0.25,0.49,0.48} % Comentarios
\definecolor{codegray}{rgb}{0.5,0.5,0.5} % Números y anotaciones
\definecolor{codepurple}{rgb}{0.58,0,0.82} % Palabras clave
\definecolor{backcolour}{rgb}{0.95,0.95,0.92} % Color de fondo

% Configuración del estilo de las celdas de código
\lstset{
    backgroundcolor=\color{backcolour},   % color de fondo; necesita que el paquete color o xcolor esté cargado
    commentstyle=\color{codegreen},       % estilo de comentarios
    keywordstyle=\color{codepurple},      % estilo de palabras clave
    numberstyle=\tiny\color{codegray},    % estilo de los números de línea
    stringstyle=\color{red},              % estilo de las cadenas de texto
    basicstyle=\ttfamily\small,           % estilo del texto básico
    breakatwhitespace=false,              % ajustes de líneas sólo en espacios en blanco
    breaklines=true,                      % ajustar las líneas si son muy largas
    captionpos=b,                         % posición de la leyenda (abajo)
    keepspaces=true,                      % preserva los espacios en el texto; útil si se usa monoespaciado
    numbers=left,                         % dónde poner los números de línea
    numbersep=5pt,                        % qué tan lejos están los números de línea del código
    showspaces=false,                     % mostrar espacios con subrayados particulares; reemplaza 'showstringspaces'
    showstringspaces=false,               % subrayar los espacios dentro de las cadenas solo
    showtabs=false,                       % mostrar tabulaciones en el código con subrayados particulares
    tabsize=2,                            % tamaños de tabulación a 2 espacios
    language=TeX,                         % lenguaje del código
    morecomment=[l]\#,                    % reconocer # como inicio de comentario en Python
    frame=single,                         % agregar un marco simple alrededor del código
    rulecolor=\color{black}               % color del marco
}

\begin{document}
%----------------------------------------------------------------------------------------
%   PORTADA
%Modificar desde aqui en adelante
%----------------------------------------------------------------------------------------
\begin{titlepage}%Inicio de la carátula, solo modificar los datos necesarios
\newcommand{\HRule}{\rule{\linewidth}{0.5mm}} 
\center 
%----------------------------------------------------------------------------------------
%	ENCABEZADO
%----------------------------------------------------------------------------------------
%----------------------------------------------------------------------------------------
%	SECCION DEL TITULO
%----------------------------------------------------------------------------------------
\begin{center}
    \textbf{\LARGE Generación y Atracción de Viajes} \\[0.5cm]
    \textbf{Felipe Vicencio y Lukas Wolff} \\
    Facultad de Ingenieria y Ciencias Aplicadas, Universidad de los Andes, Santiago de Chile.\\
    email: \href{mailto:lwolff@miuandes.cl}{lwolff@miuandes.cl}, \href{mailto:favicencio@miuandes.cl}{favicencio@miuandes.cl}
    \\
    github: \href{https://github.com/LukasWolff2002/TAREA_2_AUTITOS.git}{Link al repositorio}
\end{center}

\vspace{1cm}

\begin{center}
    \textbf{\large RESUMEN}    
\end{center}

\begin{flushleft}
    Hablar aqui del resumen del informe actual, no puede exceer las 250 palabras. Poner palabras clave, ademas de selccionar 3 palabras clave por los autores
\end{flushleft}


\vspace{1cm}

\end{titlepage}
%----------------------------------------------------------------------------------------
%	SECCION DE LA FECHA
%----------------------------------------------------------------------------------------
%{\large \textbf{\today}}\\[2cm] % El comando \today coloca la fecha del dia, y esto se actualiza con cada compilacion, en caso de querer tener una fecha estatica, reemplazar el \today por la fecha deseada

%----------------------------------------------------------------------------------------
%  INDICE
%----------------------------------------------------------------------------------------
%\newpage
%\tableofcontents
%\thispagestyle{plain} % Deshabilita el encabezado en la página del índice
%\thispagestyle{empty} % Deshabilita el número de página en la página del índice
%\newpage

%Se puede agregar un indice de figuras si es nesesario
%\newpage
%\listoffigures 
%\thispagestyle{plain} % Deshabilita el encabezado en la página del índice %
%\thispagestyle{empty}
%\newpage
%----------------------------------------------------------------------------------------
%   ACÁ EMPIEZA EL INFORME
\setcounter{page}{1} % Reinicia el contador de páginas
%----------------------------------------------------------------------------------------
%Este es el formato a seguir para los titulos de las secciones

\section{Introducción}

El estudio de los viajes es fundamental para el desarrollo de las zonas altamente pobladas, ya que permite ahorrar tiempos de viaje, ahorros en combustible e incluso, mejorar el bienestar de las personas. Para poder generar estos analisis, es nesesario recaudar data real sobre el comportamiento de las personas como su tipo de transporte, origne y destino, horas de viajes, entre otros.
\\ \\
El problema de estos estudios radica en la cantidad de datos que se nesesitan para poder realizar un analisis completo, por lo tanto, se buscan hacer estimaciones mediante una regrecion lineal para poder representar a una poblacion de la manera mas eficaz posible.
\\ \\
En el presente informe, se buscara analizar el numero de viajes de la comuna de Las Condes, en base a la data obtenida de la encuesta ESI, para los años 2012, 2017 y 2023. Para esto, se utilizara un modelo de regresion lineal para poder estimar el numero de viajes de la comuna en base a distintos factores que pueden influir en la cantidad de viajes.
\\ \\
Finalmente, se analizara y discutira sobre la eficiencia del modelo, la precision de los datos obtenidos y los factores que pueden influir en la cantidad de viajes de la comuna de Las Condes.

\section{Contenido}

\subsection{Matriz Origen Destino}

Para la representacion de todos los viajes de la red, se puede utilizar la matriz origen destino, la cual se puede representar de la siguiente manera:

\begin{equation}
    \begin{bmatrix}
        O_{1}D_{1} & O_{1}D_{2} & \cdots & O_{1}D_{n} \\
        O_{2}D_{1} & O_{2}D_{2} & \cdots & O_{2}D_{n} \\
        \vdots & \vdots & \ddots & \vdots \\
        O_{n}D_{1} & O_{n}D_{2} & \cdots & O_{n}D_{n}
    \end{bmatrix}
\end{equation}

El problema es que normalmente no se tienen suficientes dato para poder completar de manera satisfactoria la matriz origen destino, por lo que se deben utilizar modelos de estimacion de viajes para poder completarla.
\\ \\
De esta manera, es importante analizar los distintos factores que pueden influir en los viajes de las personas.

\subsection{Regresión Lineal}

En base a los datos obtenidos por distintas encuetas, se busca obtener una relacion que explique el comportamiento de la poblacion:

\begin{equation}
    Y = \alpha + \beta_{1}X_{1} + \beta_{2}X_{2} + \cdots + \beta_{n}X_{n} + \epsilon
\end{equation}

Lo cual se puede extrapolar a:

\begin{equation}
    Y = \alpha + \vec{\beta}\vec{X} + \epsilon
\end{equation}

Donde Y corresponde al numero de viajes y $\vec{X}$ corresponden a los distintos factores que pueden afectar a la cantidad de viajes. En el caso de este informe, se utilizaran las siguientes variables X:

\begin{center}
    $X_i = log_{10}(IPCH)$, donde IPCH correspone al ingreso per capita por hogar\\
    $X_p$ = numero de personas con edad $\epsilon$ [0, 5]\\
    $X_e$ = numero de personas con edad $\epsilon$ [6, 22]\\
    $X_t$ = numero de personas con edad $\epsilon$ [23, 62]\\
    $X_j$ = numero de personas con edad $\epsilon$ [63, 79]\\
\end{center}

De esta forma, el model ode regrecion lineal queda de la siguiente forma:

\begin{equation}
    Y_i = -2.1723 + 0.3792X_i + 0.6221X_p + 1.0065X_e + 0.4302X_t + 0.1614X_j
    \label{eq:regrecion}
\end{equation}

Es importante mencionar que los coeficientes $\vec{\beta}$ y $\alpha$ fueron entregados por el enunciado, donde el modelo fue calibrado segun los datos del 2012.

\section{Obtencion Resultados Bases de Datos}

Para el analisis y  discucion de la regracion lineal, se utilizara la base de datos ESI \textbf{\cite{esi}}, la cual contiene toda la informacion nesesaria para aplicar en el modelo.
\\ \\
De esta forma, en primer lugar se realizo un filtrado de la comuna Las COndes, ademas de calcular los distintos factores y promediarlos para los años 2012, 2017 y 2023.

\begin{table}[H]
    \centering
    \caption{Coeficientes promediados}
    \vspace{0.2cm}
    \begin{tabular}{|c|c|c|c|c|c|c|c|}
        \hline
        Año & Numero Hogares & $X_i$ & $X_p$ & $X_e$ & $X_t$ & $X_j$ & Tamaño hogar \\
        \hline
        2012 & 271 &5.730 & 0.258 & 0.771 & 1.978 & 0.303 & 3.16\\
        2017 & 272 &5.773 & 0.243 & 0.691 & 1.978 & 0.320 & 3.09\\
        2023 & 187 &5.934 & 0.160 & 0.668 & 1.775 & 0.299 & 2.60\\
        \hline
    \end{tabular}
    \vspace{0.2cm}
    \\Fuente: Elaboracion propia a partir de los datos de encueta ESI \textbf{\cite{esi}}
\end{table}

Posteriormente, se consulto la encuesta de Estimacion y Proyeccion, de esta forma se tendran los datos reales sobre la poblacion, para poder comparar con la muestra de la encuesta ESI.

\begin{table}[H]
    \centering
    \caption{Valores Poblacion}
    \vspace{0.2cm}
    \begin{tabular}{|c|c|c|c|c|c|}
        \hline
        \diagbox{Año}{Coeficiente} & $X_p$ & $X_e$ & $X_t$ & $X_j$ & Total hogares \\
        \hline
        2012 & 0.219 & 0.705 & 1.751 & 0.483 & 90304 \\
        2017 & 0.204 & 0.611 & 1.736 & 0.537 & 99582 \\
        2023 & 0.156 & 0.457 & 1.475 & 0.510 & 131224 \\
        \hline
    \end{tabular}
    \vspace{0.2cm}
    \\Fuente: Elaboracion propia a partir de los datos de Estimación y Proyección 2002 - 2035
\end{table}

Conociendo el total de hogares de la poblacion, es posible calcular el numero de viajes totales obtenidos a partir de la regracion lineal:

\begin{table}[H]
    \centering
    \caption{Estimacion de viajes totales}
    \vspace{0.2cm}
    \begin{tabular}{|c|c|c|}
        \hline
        Año & $Y_i$ & Viajes Totales \\
        \hline
        2012 & 1.84 & 192554.16\\
        2017 & 1.77 & 208872.39\\
        2023 & 1.66 & 195891.62\\
        \hline
    \end{tabular}
    \vspace{0.2cm}
    \\ Fuente: Elaboracion propia a partir de los resultados de la ecuacion \ref{eq:regrecion}
\end{table}

\section{Discusiones}

\subsection{Inflacion}

Un punto clave a considerar sobre el comportamiento de las personas es el valor y peso que tiene la moneda en el tiempo, de esta manera, es nesesario considerar la inflacion, y el aumento del precio en los distintos insumos del transporte.

\begin{table}[H]
    \centering
    \caption{Inflación en Chile}
    \vspace{0.2cm}
    \begin{tabular}{|c|c|}
        \hline
        Año & Inflacion \\
        \hline
        2012 - 2017 & 18,2\% \\
        2012 - 2023 & 61,5\% \\
        \hline
    \end{tabular}
    \vspace{0.2cm}
    \\Fuente: \textbf{\cite{ipc}}
\end{table}

La regrecion lineal aplicada fue calibrada segun los datos del 2012, por lo tanto, se puede ajustar $X_i$ segun la inflacion para obtener un parametro de viajes mas acertado:

%Donde, algunos de los insumos relacionados con el transporte que han aumentado de precio son la bencina y el pesaje de transporte publico (usaremos el metro) estos se encuentran en la tabla \textbf{\ref{Insumos}}. Las variaciones porcentuales fueron calculadas sobre el año 2012

%Considerando los resultados obtenidos, se puede ajustar el parametro Xi para ajustar la regrecion a los distintos años:

\begin{table}[H]
    \centering
    \caption{Ajuste de modelo por inflacion}
    \vspace{0.2cm}
    \begin{tabular}{|c|c|c|c|}
        \hline
        Año & $X_i$ & $Y_i$ & Viajes Totales \\
        \hline
        2012 & 5.730 & 1.84 & 166159.36\\
        2017 & 5.700 & 1.74 & 173272.68\\
        2023 & 5.725 & 1.58 & 207333.92\\
        \hline
    \end{tabular}
    \vspace{0.2cm}
    \\Fuente: Elaboracion propia
\end{table}


\subsection{Precisión de Resultados}

Analizando los datos presentados junto con las bases de datos, se estima que la precisión de los resultados obtenidos es fiable, ya que 
las fuentes de donde se recopiló la información fueron directamente de las páginas web 
gubernamentales y municipales, así como también de los agentes encargados de análisis estadísticos sociales, como el INE.

Si bien se evidencia una diferencia de los parámetros $X_p$, $X_e$, $X_t$ y $X_j$, se debe a la comparación de datos muestrales
con poblacionales, los cuales se compararon estadísticamente (ver Tabla \textbf{\ref{Precision_Resultados}}).

Se puede apreciar que los resultados obtenidos poseen una desviación estándar insignificante, por lo que se 
consideran representativos para cada coeficiente. Además, se realizó un análisis del estadístico t para 
comparar la media muestral con la media poblacional. Debido a que \(\left| p_{value} \right| < \left| t_{value} \right|\), existe una correlación entre las medias, 
por lo que se puede afirmar que los resultados obtenidos son precisos y descriptivos.

\subsection{Analisis}

Considerando todo lo expuesto anteriormente, se puede observar que el modelo de regrecion lineal sin ser ajustado es fiable con cierto margen de error, donde se concluye que el numero de viajes disminuye por cada casa lo cual tiene sentido, ya que luego de analizar la inflacion y el valor de los distintos insumos de transporte (ver Tabla \textbf{\ref{Insumos}}), se puede concluir que el valor de un viaje no aumenta con mayor proporcion que la inflacion, por lo tanto el valor de este aumenta. Ademas, se puede observar que el tamaño de las familias disminuye, lo cual influye en el numero de viajes. Finalmente, es nesesario considerar el efecto que tuvo la pandemia sobre las personas, donde muchos trabajos comenzaron a ser virtuales o semi presenciales, lo cual afecta el transporte en familias, sobre todo las de menor tamaño.
\\ \\
Se observa que al calibrar el modelo segun la inflacion, los viajes por hogar bajan aun mas, lo cual se relaciona con la perdida de valor que tiene el dinero y no a sido proporcional a los aumentos de sueldos (ver Tabla \ref{VariacionIPCH}).
\\ \\
Finalmente, haciendo un analisis estadistico de los datos, se concluye que los datos se pueden considerar representativos y precisos, dado la procedencia de estos y su correlacion muestral con poblacional según el estadístico T Student.

\section{Conclusión}

En base a los resultados obtenidos, se puede concluir que el modelo 
de regresión lineal es fiable para estimar el numero de viajes de la 
comuna de Las Condes, sin embargo, es nesesario ajustar el modelo según 
la inflación para obtener resultados mas precisos. Además, se concluye 
que el número de viajes disminuye año a año, junto con el tamaño familiar. 
Con esto, se puede decir que fue un trabajo exitoso, ya que se midió y 
analizó la información de manera correcta, atribuyendo respaldo a la precisión 
de los resultados y validéz a la regresión lineal.

\section{Anexos}

\subsection{Tablas Inflación}

\begin{table}[H]
    \centering
    \caption{Aumento de indicadores economicos}
    \vspace{0.2cm}
    \begin{tabular}{|c|c|c|c|c|}
        \hline
        Año & UF (Pesos) & $\Delta$ \% UF & Valor Dolar & $\Delta$ \% Dolar\\
        \hline
        2012 & 22.296,19 & - & 501.34 & - \\
        2017 & 26.348,83 & 18.2\% & 661.19 & 31.88 \\
        2023 & 35.122,26 & 57.6\% & 826.34 & 64.82 \\
        \hline
    \end{tabular}
    \vspace{0.2cm}
    \\Fuente: \textbf{\cite{sii}}
\end{table}

\begin{table}[H]
    \centering
    \caption{Aumento de precio de insumos}
    \vspace{0.2cm}
    \begin{tabular}{|c|c|c|c|c|}
        \hline
        Año &  Bencina 95 (USD) & $\Delta$ \% Bencina & Metro (CLP) & $\Delta$ \% Metro\\
        \hline
        2012 &  1.56  & - & 610 & - \\
        2017 &  1.15 & -26.28 & 660 & 8.19\\
        2023 &  1.64 & 5.13 & 730 & 19.67\\
        \hline
    \end{tabular}
    \label{Insumos}
    \vspace{0.2cm}
    \\Fuente: \textbf{\cite{tradingeconomics}, \cite{bcentral}}
\end{table}

\begin{table}[H]
    \centering
    \caption{Variacion IPCH ajustado por inflacion}
    \vspace{0.2cm}
    \begin{tabular}{|c|c|c|}
        \hline
        Año & $X_i$ & $\Delta$ \% \\
        \hline
        2012 & 5.730 & - \\
        2017 & 5.700 & -0.52 \\
        2023 & 5.725 & -0.09 \\
        \hline
    \end{tabular}
    \vspace{0.2cm}
    \label{VariacionIPCH}
    \\Fuente: Elaboracion propia
\end{table}

\subsection{Presicion de Datos} \label{sec:Precision}

\begin{table}[H]
    \caption{Análisis estadístico para la precisión de resultados.}
    \vspace{0.2cm}
    \centering
    \begin{tabular}{|c|c|c|c|}
    \hline
    \textbf{Coeficiente} & \textbf{Estadístico} & \textbf{Población} & \textbf{Muestra} \\ \hline
    $X_p$ & \multicolumn{1}{|c|}{\(\mu\)} & 0.193 & 0.220 \\ \cline{2-4}
    & \multicolumn{1}{|c|}{\(\sigma\)} & 0.027 & 0.053 \\ \cline{2-4}
    & \multicolumn{1}{|c|}{Valor p} & \multicolumn{2}{|c|}{1} \\ \cline{2-4}
    & \multicolumn{1}{|c|}{Valor t} & \multicolumn{2}{|c|}{-26.09} \\ \hline
    
    $X_e$ & \multicolumn{1}{|c|}{\(\mu\)} & 0.591 & 0.710 \\ \cline{2-4}
    & \multicolumn{1}{|c|}{\(\sigma\)} & 0.102 & 0.054 \\ \cline{2-4}
    & \multicolumn{1}{|c|}{Valor p} & \multicolumn{2}{|c|}{1} \\ \cline{2-4}
    & \multicolumn{1}{|c|}{Valor t} & \multicolumn{2}{|c|}{-31.52} \\ \hline
    
    $X_t$ & \multicolumn{1}{|c|}{\(\mu\)} & 1.654 & 1.910 \\ \cline{2-4}
    & \multicolumn{1}{|c|}{\(\sigma\)} & 0.126 & 0.117 \\ \cline{2-4}
    & \multicolumn{1}{|c|}{Valor p} & \multicolumn{2}{|c|}{1} \\ \cline{2-4}
    & \multicolumn{1}{|c|}{Valor t} & \multicolumn{2}{|c|}{-55.84} \\ \hline
    
    $X_j$ & \multicolumn{1}{|c|}{\(\mu\)} & 0.510 & 0.307 \\ \cline{2-4}
    & \multicolumn{1}{|c|}{\(\sigma\)} & 0.021 & 0.011 \\ \cline{2-4}
    & \multicolumn{1}{|c|}{Valor p} & \multicolumn{2}{|c|}{0} \\ \cline{2-4}
    & \multicolumn{1}{|c|}{Valor t} & \multicolumn{2}{|c|}{261.09} \\ \hline
    \end{tabular}
    \label{Precision_Resultados}
    \vspace{0.2cm}
    \\Fuente: Elaboración propia.
\end{table}

\subsection{Tablas Datos Encuesta EP} \label{sec:Datos_EP}

\begin{table}[H]
    \centering
    \caption{Rango etario en Las Condes año 2012}
    \vspace{0.2cm}
    \begin{tabular}{|c|c|c|c|c|c|}
        \hline
        \textbf{Rango etario} & \textbf{Hombres} & \textbf{Mujeres} & \textbf{Total} & \textbf{\% Hombres} & \textbf{\% Mujeres} \\ \hline
        0-5 & 10484 & 9377 & 19861 & 52.79\% & 47.21\% \\ \hline
        6-22 & 33329 & 30366 & 63695 & 52.32\% & 47.67\% \\ \hline
        23-62 & 75938 & 82177 & 158115 & 48.02\% & 51.97\% \\ \hline
        63-80 & 17595 & 26095 & 43690 & 40.27\% & 59.72\% \\ \hline
    \end{tabular}
    \label{Cuadro 4}
    \vspace{0.2cm}
    \\Fuente: Elaboración propia.
\end{table}

\begin{table}[H]
    \centering
    \caption{Rango etario en Las Condes año 2017}
    \vspace{0.2cm}
    \begin{tabular}{|c|c|c|c|c|c|}
        \hline
        \textbf{Rango etario} & \textbf{Hombres} & \textbf{Mujeres} & \textbf{Total} & \textbf{\% Hombres} & \textbf{\% Mujeres} \\ \hline
        0-5 & 10672 & 9692 & 20364 & 52.41\% & 47.59\% \\ \hline
        6-22 & 31833 & 29028 & 60861 & 52.31\% & 47.69\% \\ \hline
        23-62 & 84879 & 88035 & 172914 & 49.08\% & 50.91\% \\ \hline
        63-80 & 21753 & 31816 & 53569 & 40.61\% & 59.39\% \\ \hline
    \end{tabular}
    \label{Cuadro 5}
    \vspace{0.2cm}
    \\Fuente: Elaboración propia.
\end{table}

\begin{table}[H]
    \centering
    \caption{Rango etario en Las Condes año 2023}
    \vspace{0.2cm}
    \begin{tabular}{|c|c|c|c|c|c|}
        \hline
        \textbf{Rango etario} & \textbf{Hombres} & \textbf{Mujeres} & \textbf{Total} & \textbf{\% Hombres} & \textbf{\% Mujeres} \\ \hline
        0-5 & 10769 & 9758 & 20527 & 52.46\% & 47.53\% \\ \hline
        6-22 & 31416 & 28678 & 60094 & 52.27\% & 47.72\% \\ \hline
        23-62 & 96289 & 97341 & 193630 & 49.72\% & 50.27\% \\ \hline
        63-80 & 27825 & 39107 & 66932 & 41.57\% & 58.42\% \\ \hline
    \end{tabular}
    \label{Cuadro 6}
    \vspace{0.2cm}
    \\Fuente: Elaboración propia.
\end{table}

\bibliographystyle{plainnat}  % Estilo de la bibliografía (puedes cambiarlo a otro si prefieres)
\bibliography{referencias.bib}  % Nombre del archivo .bib

\end{document}
