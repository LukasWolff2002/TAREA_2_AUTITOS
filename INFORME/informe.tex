%NO MODIFICAR ESTA SECCION!
\documentclass[12pt]{article} % Define la clase del documento, en este caso, un artículo
\usepackage{mathptmx}
\usepackage[letterpaper,margin=3cm]{geometry} % Configura el tamaño del papel y los márgenes del documento
\usepackage{graphicx} % Permite la inserción de imágenes
\usepackage[spanish]{babel}% Activar esta configuración para informes en español, ajusta el idioma del documento
\usepackage[usenames]{color} % Permite el uso de colores definidos por nombre en el documento
\usepackage{hyperref} % Habilita enlaces y referencias dentro del documento
\hypersetup{colorlinks=true, linkcolor = black, citecolor= black} % Configura el color de los enlaces y citas
\usepackage{booktabs} % Proporciona comandos para crear tablas de alta calidad
\usepackage{natbib} % Permite el uso de citas y referencias bibliográficas con diferentes estilos
\usepackage{tikz} % Permite la creación de gráficos y diagramas vectoriales directamente en LaTeX
\usepackage{float} % Para controlar la posición de los elementos flotantes, como imágenes, con la opción [H]
\bibliographystyle{agsm} % Define el estilo de citas y bibliografía (en este caso, el estilo AGSM)
\usepackage{diagbox} % Permite crear celdas con líneas diagonales en tablas
\usepackage{listings} % Permite la inclusión y formateo de código fuente en el documento
\usepackage{xcolor} % Paquete para definir y usar colores en el documento
\usepackage{parskip} % Añade espacio entre párrafos en lugar de sangrías
\usepackage{fancyhdr} % Permite personalizar encabezados y pies de página
\usepackage{amsmath} % Proporciona una amplia variedad de entornos y comandos matemáticos
\usepackage{microtype}  % Mejora la justificación del texto

\geometry{
    paperwidth=21.6cm,  % Ancho del papel
    paperheight=27.9cm,  % Largo del papel
    left=3cm,  % Margen izquierdo
    right=2cm,  % Margen derecho
}

\pagestyle{fancy} % Usa el estilo fancyhdr
\fancyhf{} % Borra todos los encabezados y pies de página
\renewcommand{\headrulewidth}{0pt}
\renewcommand{\footrulewidth}{0pt} % Desactiva la línea horizontal predeterminada en el pie
\setlength{\headheight}{1.5cm} % Ajusta la altura del encabezado para hacer espacio para la línea
\fancyhead[L]{\raisebox{0.20cm}{\textbf{Sistemas de Transporte}}} % Añade el texto en la parte izquierda del encabezado, subiéndolo ligeramente
\fancyhead[R]{\raisebox{0.1cm}{\includegraphics[width=0.25\linewidth]{LOGO_UNIVERSIDAD.jpg}}} % Añade la imagen en la parte derecha del encabezado y súbela un poco
\fancyhead[C]{\rule{\textwidth}{0.6pt}} % Añade una línea horizontal superior centrada
\fancyfoot[C]{\rule{\textwidth}{0.6pt}} % Añade una línea horizontal en el pie de página centrada
\fancyfoot[R]{\raisebox{-1.5\baselineskip}{\thepage}} % Coloca el número de página a la derecha, con suficiente espacio debajo de la línea
\geometry{top=2.5cm, bottom=2.5cm} % Ajusta los márgenes superior e inferior

% Definición de colores al estilo Visual Studio Code
\definecolor{codegreen}{rgb}{0.25,0.49,0.48} % Comentarios
\definecolor{codegray}{rgb}{0.5,0.5,0.5} % Números y anotaciones
\definecolor{codepurple}{rgb}{0.58,0,0.82} % Palabras clave
\definecolor{backcolour}{rgb}{0.95,0.95,0.92} % Color de fondo

% Configuración del estilo de las celdas de código
\lstset{
    backgroundcolor=\color{backcolour},   % color de fondo; necesita que el paquete color o xcolor esté cargado
    commentstyle=\color{codegreen},       % estilo de comentarios
    keywordstyle=\color{codepurple},      % estilo de palabras clave
    numberstyle=\tiny\color{codegray},    % estilo de los números de línea
    stringstyle=\color{red},              % estilo de las cadenas de texto
    basicstyle=\ttfamily\small,           % estilo del texto básico
    breakatwhitespace=false,              % ajustes de líneas sólo en espacios en blanco
    breaklines=true,                      % ajustar las líneas si son muy largas
    captionpos=b,                         % posición de la leyenda (abajo)
    keepspaces=true,                      % preserva los espacios en el texto; útil si se usa monoespaciado
    numbers=left,                         % dónde poner los números de línea
    numbersep=5pt,                        % qué tan lejos están los números de línea del código
    showspaces=false,                     % mostrar espacios con subrayados particulares; reemplaza 'showstringspaces'
    showstringspaces=false,               % subrayar los espacios dentro de las cadenas solo
    showtabs=false,                       % mostrar tabulaciones en el código con subrayados particulares
    tabsize=2,                            % tamaños de tabulación a 2 espacios
    language=TeX,                         % lenguaje del código
    morecomment=[l]\#,                    % reconocer # como inicio de comentario en Python
    frame=single,                         % agregar un marco simple alrededor del código
    rulecolor=\color{black}               % color del marco
}

\begin{document}
%----------------------------------------------------------------------------------------
%   PORTADA
%Modificar desde aqui en adelante
%----------------------------------------------------------------------------------------
\begin{titlepage}%Inicio de la carátula, solo modificar los datos necesarios
\newcommand{\HRule}{\rule{\linewidth}{0.5mm}} 
\center 
%----------------------------------------------------------------------------------------
%	ENCABEZADO
%----------------------------------------------------------------------------------------
%----------------------------------------------------------------------------------------
%	SECCION DEL TITULO
%----------------------------------------------------------------------------------------
\begin{center}
    \textbf{\LARGE Generación y Atracción de Viajes} \\[0.5cm]
    \textbf{Felipe Vicencio y Lukas Wolff} \\
    Facultad de Ingenieria y Ciencias Aplicadas, Universidad de los Andes, Santiago de Chile.\\
    email: favicencio@miuandes.cl , lwolff@miuandes.cl
\end{center}

\vspace{1cm}

\begin{center}
    \textbf{\large RESUMEN}    
\end{center}

\begin{flushleft}
    Hablar aqui del resumen del informe actual, no puede exceer las 250 palabras. Poner palabras clave, ademas de selccionar 3 palabras clave por los autores
\end{flushleft}


\vspace{1cm}

\end{titlepage}
%----------------------------------------------------------------------------------------
%	SECCION DE LA FECHA
%----------------------------------------------------------------------------------------
%{\large \textbf{\today}}\\[2cm] % El comando \today coloca la fecha del dia, y esto se actualiza con cada compilacion, en caso de querer tener una fecha estatica, reemplazar el \today por la fecha deseada

%----------------------------------------------------------------------------------------
%  INDICE
%----------------------------------------------------------------------------------------
%\newpage
%\tableofcontents
%\thispagestyle{plain} % Deshabilita el encabezado en la página del índice
%\thispagestyle{empty} % Deshabilita el número de página en la página del índice
%\newpage

%Se puede agregar un indice de figuras si es nesesario
%\newpage
%\listoffigures 
%\thispagestyle{plain} % Deshabilita el encabezado en la página del índice %
%\thispagestyle{empty}
%\newpage
%----------------------------------------------------------------------------------------
%   ACÁ EMPIEZA EL INFORME
\setcounter{page}{1} % Reinicia el contador de páginas
%----------------------------------------------------------------------------------------
%Este es el formato a seguir para los titulos de las secciones

\section{Introducción}

Hablar de las importancias y aplicaciones de estos modelos en la estimacion de viajes como para:

\begin{itemize}
    \item Planificación de transporte
    \item Diseño de infraestructura
    \item Evaluación de proyectos
    \item Análisis de políticas
    \item Estudios de impacto
    \item Estudios de demanda
    \item Estudios de oferta
    \item Estudios de accesibilidad
    \item Estudios de congestión
    \item Estudios de externalidades
\end{itemize}

\section{Contenido}

\newpage
\section{Ecuaciones, Tablas y Figuras}

\subsection{Matriz Origen Destino}

Para la representacion de todos los viajes de la red, se puede utilizar la matriz origen destino, la cual se puede representar de la siguiente manera:

\begin{equation}
    \begin{bmatrix}
        O_{1}D_{1} & O_{1}D_{2} & \cdots & O_{1}D_{n} \\
        O_{2}D_{1} & O_{2}D_{2} & \cdots & O_{2}D_{n} \\
        \vdots & \vdots & \ddots & \vdots \\
        O_{n}D_{1} & O_{n}D_{2} & \cdots & O_{n}D_{n}
    \end{bmatrix}
\end{equation}

El problema es que normalmente no se tienen suficientes dato para poder completar de manera satisfactoria la matriz origen destino, por lo que se deben utilizar modelos de estimacion de viajes para poder completarla.
\\ \\
De esta manera, es importante analizar los distintos factores que pueden influir en los viajes de las personas.

\subsection{Análisis Regresión Lineal}

En base a los datos obtenidos por distintas encuetas, se busca obtener una relacion que explique el comportamiento de la poblacion:

\begin{equation}
    Y = \alpha + \beta_{1}X_{1} + \beta_{2}X_{2} + \cdots + \beta_{n}X_{n} + \epsilon
\end{equation}

Lo cual se puede extrapolar a:

\begin{equation}
    Y = \alpha + \vec{\beta}\vec{X} + \epsilon
\end{equation}

Donde Y corresponde al numero de viajes y $\vec{X}$ corresponden a los distintos factores que pueden afectar a la cantidad de viajes. En el caso de este informe, se utilizaran las siguientes variables X:

\begin{center}
    $X_i = log_{10}(IPCH)$, donde IPCH correspone al ingreso per capita por hogar\\
    $X_p$ = numero de personas con edad $\epsilon$ [0, 5]\\
    $X_e$ = numero de personas con edad $\epsilon$ [6, 22]\\
    $X_t$ = numero de personas con edad $\epsilon$ [23, 62]\\
    $X_j$ = numero de personas con edad $\epsilon$ [63, 79]\\
\end{center}

De esta forma, el model ode regrecion lineal queda de la siguiente forma:

\begin{equation}
    Y_i = -2.1723 + 0.3792X_i + 0.6221X_p + 1.0065X_e + 0.4302X_t + 0.1614X_j
\end{equation}

Es importante mencionar que los coeficientes $\vec{\beta}$ y $\alpha$ fueron entregados por el enunciado.

\section{Resultados Bases de Datos}
DESPUES NO PONER LAS SECCIONES DE ESTA FORMA.

\subsection{Resultados 2.1}

\begin{table}[H]
    \centering
    \begin{tabular}{|c|c|}
        \hline
        Año & Numero Hogares \\
        \hline
        2012 & 271 \\
        2017 & 272 \\
        2023 & 187 \\
        \hline
    \end{tabular}
    \caption{Matriz de estimacion de viajes}
    Fuente: Elaboracion propia a partir de los datos de encueta ESI
\end{table}

\begin{table}[H]
    \centering
    \begin{tabular}{|c|c|c|c|c|c|}
        \hline
        \diagbox{Año}{Coeficiente} & $X_i$ & $X_p$ & $X_e$ & $X_t$ & $X_j$ \\
        \hline
        2012 & 5.730 & 0.258 & 0.771 & 1.978 & 0.303 \\
        2017 & 5.773 & 0.243 & 0.691 & 1.978 & 0.320 \\
        2023 & 5.934 & 0.160 & 0.668 & 1.775 & 0.299 \\
        \hline
    \end{tabular}
    \caption{Matriz de estimacion de viajes}
    Fuente: Elaboracion propia a partir de los datos de encueta ESI
\end{table}

\subsection{Resultados 2.2}
A continuación se presentarán los resultaos obtenidos de la base de datos ``Estimación y Proyección 2002-2035``. Se agruparon las personas por grupo etario en la comuna de Las Condes para los años 2012, 2017 y 2023 y con el total de hogares se obtuvieron los siguientes parámetros.

\begin{table}[H]
    \centering
    \caption{Valor esperado de personas por grupo etario}
    \begin{tabular}{|c|c|c|c|c|c|}
        \hline
        \diagbox{Año}{Coeficiente} & $X_p$ & $X_e$ & $X_t$ & $X_j$ & Total hogares \\
        \hline
        2012 & 0.189 & 0.608 & 1.511 & 0.417 & 104649 \\
        2017 & 0.172 & 0.515 & 1.465 & 0.453 & 118007 \\
        2023 & 0.173 & 0.509 & 1.641 & 0.567 & 118007 \\
        \hline
    \end{tabular}
    \\Fuente: Elaboracion propia a partir de los datos de Estimación y Proyección 2002 - 2035
\end{table}

Los datos anteriores representan la cantidad de personas por rango etario presente en un hogar para cada año. 
Se realizó el análisis con datos poblacionales, es decir, con el total de hogares inscritos en el año. Se supuso la misma
cantidad de hogares para los años 2017 y 2023 debido a la falta de información proporcionada

\begin{table}[h!]
    \centering
    \caption{Rango etario en Las Condes año 2012}
    \begin{tabular}{|c|c|c|c|c|c|}
    \hline
    \textbf{Rango etario} & \textbf{Hombres} & \textbf{Mujeres} & \textbf{Total} & \textbf{\% Hombres} & \textbf{\% Mujeres} \\ \hline
    0-5 & 10484 & 9377 & 19861 & 52.79\% & 47.21\% \\ \hline
    6-22 & 33329 & 30366 & 63695 & 52.32\% & 47.67\% \\ \hline
    23-62 & 75938 & 82177 & 158115 & 48.02\% & 51.97\% \\ \hline
    63-80 & 17595 & 26095 & 43690 & 40.27\% & 59.72\% \\ \hline
    \end{tabular}
    \label{Cuadro 4}
    \\Fuente: Elaboración propia.
\end{table}

\begin{table}[h!]
    \centering
    \caption{Rango etario en Las Condes año 2017}
    \begin{tabular}{|c|c|c|c|c|c|}
    \hline
    \textbf{Rango etario} & \textbf{Hombres} & \textbf{Mujeres} & \textbf{Total} & \textbf{\% Hombres} & \textbf{\% Mujeres} \\ \hline
    0-5 & 10672 & 9692 & 20364 & 52.41\% & 47.59\% \\ \hline
    6-22 & 31833 & 29028 & 60861 & 52.31\% & 47.69\% \\ \hline
    23-62 & 84879 & 88035 & 172914 & 49.08\% & 50.91\% \\ \hline
    63-80 & 21753 & 31816 & 53569 & 40.61\% & 59.39\% \\ \hline
    \end{tabular}
    \label{Cuadro 5}
    \\Fuente: Elaboración propia.
\end{table}

\begin{table}[h!]
    \centering
    \caption{Rango etario en Las Condes año 2023}
    \begin{tabular}{|c|c|c|c|c|c|}
    \hline
    \textbf{Rango etario} & \textbf{Hombres} & \textbf{Mujeres} & \textbf{Total} & \textbf{\% Hombres} & \textbf{\% Mujeres} \\ \hline
    0-5 & 10769 & 9758 & 20527 & 52.46\% & 47.53\% \\ \hline
    6-22 & 31416 & 28678 & 60094 & 52.27\% & 47.72\% \\ \hline
    23-62 & 96289 & 97341 & 193630 & 49.72\% & 50.27\% \\ \hline
    63-80 & 27825 & 39107 & 66932 & 41.57\% & 58.42\% \\ \hline
    \end{tabular}
    \label{Cuadro 6}
    \\Fuente: Elaboración propia.
\end{table}



\subsection{Resultados 2.3}

Nesecito el numero total de hogares
numero total hogares 2012 = 314
numero total hogares 2017 = 326
numero total hogares 2023 = 258
sacado de las bases de datos

\section{Discusiones}

\begin{itemize}
    \item ¿Qué tan precisos cree que son cada uno de los resultados? ¿Por qué?
    \item ¿Es suficiente la información del modelo de regresión lineal obtenido?
    \item ¿Qué cosas han cambiado entre el 2012 y el 2023 que puedan haber tenido un impacto significativo en la generación de viajes, más allá de las incluidas en el modelo?
\end{itemize}

%----------------------------------------------------------------------------------------
\end{document}
